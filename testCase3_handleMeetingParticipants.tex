\begin{testCase3_handleMeetingParticipants}{\textwidth}{ |X|X| }
\hline
\rowcolor{Gray}
Test ID & 3 \\ \hline
Test item & Manage meeting participants and respond to invites. \\ \hline
Approach & Appointment-creater opens an appointment \\ \hline
Item pass/ fail criteria & All registered members are available to invite. \\ \hline
Input data & 
\begin*{itemize}
	\item Participants the user wants to invite. 
	\item Groups the user wants to invite. 
\end{itemize}\\ \hline
Expected result & 
\begin{enumerate}
	\item Participants and groups the user invited sees the new appointment in their calendar.
	\item Responds are shown in the appointment.
	\item A user can choose to hide rejected appointments
\end{enumerate} \\ \hline
Testing task &
\begin{task steps}
	\item 1. The user initiate edit or add appointment
	\item 2. A popup window or panel is shown. 
	\item 3. The user select participants from the user list and groups and invites. 
	\item 4. Login as one of the invited users. 
	\item 5. The appointment is shown in the calendar
	\item 6. Reject the invite and hide the appointment
	\item 7. Login as one of the group members invited
	\item 8. The appointment is shown in the calendar
	\item 9. Accept the invite.
	\item 10. Login as the user in step 3. Check that the updated status from the participants are shown. 
\end{task steps}	\\ \hline
Necessary environmental requirements & More than two users exist, and one group.   \\ \hline
References to user scenario, use case, sequence diagrams, and overall class diagram & None
Any dependency between this test and the other tests defined & Login has to work, add or edit appointment  \\ \hline

\end{testCase3_handleMeetingParticipants}
