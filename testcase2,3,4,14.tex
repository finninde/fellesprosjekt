\documentclass[a4paper, 10pt]{article}
\usepackage[utf8x]{inputenc}
\usepackage[norsk]{babel}
\usepackage{natbib}
\usepackage{graphicx}
\usepackage[T1]{fontenc}
\usepackage{amsmath}
\usepackage{mathtools}
\usepackage{tabularx}
\usepackage{color, colortbl}

\definecolor{Gray}{gray}{0.9}

\begin{document}



\subsection{Test case 2:Add Appointment}
\begin{tabularx}{\textwidth}{ |X|X| }
\hline
\rowcolor{Gray}
Test ID & 2 \\ \hline
Test item & Add an appointment \\ \hline
Approach & User initiate the add an appointment action \\ \hline
Item pass/ fail criteria & An appointment is created and shown in the participants calendars, none of the other users calendar\\ \hline
Input data & 
\begin*{inputData}
	\item Start-date and time
	\item End-date and time
	\item Description
	\item (Optional) A list of participants
	\item (Optional) A location or meeting room. 
\end{inputData}\\ \hline
Expected result & 
\begin{enumerate}
	\item The appointment with correct timeframe, description and other info is shown in the invited users calendars.
\end{enumerate} \\ \hline
Testing task &
\begin{task steps}
	\item 1. The user initiate add appointment
	\item 2. A popup window or panel is shown. 
	\item 3. A user set a appointment name. 
	\item 4. The user selects: start-date and time, end-date and time. 
	\item 5. The user adds a description
	\item 6. The user select participants
	\item 7. The user select a location
	\item 8. The user saves.
	\item 9. The appointment is shown in the calendars
	\item 10. Login as one of the participants and make sure the new appointment is shown and correct.  
	\item 11. Login as a user who isn't invited, check that the newly created appointment doesn't show. 
\end{task steps}	\\ \hline
Necessary environmental requirements & More than two user exist.   \\ \hline
Any dependency between this test and the other tests defined & Login has to work  \\ \hline

\end{tabularx}
\subsection{Test case 3 Manage meeting participants and respond to invites}

\begin{tabularx}{\textwidth}{ |X|X| }
\hline
\rowcolor{Gray}
Test ID & 3 \\ \hline
Test item & Manage meeting participants and respond to invites. \\ \hline
Approach & Appointment-creater opens an appointment \\ \hline
Item pass/ fail criteria & All registered members are available to invite. \\ \hline
Input data & 
\begin{inputData}
	\item Participants the user wants to invite. 
	\item Groups the user wants to invite. 
\end{inputData}\\ \hline
Expected result & 
\begin{enumerate}
	\item Participants and groups the user invited sees the new appointment in their calendar.
	\item Responds are shown in the appointment.
	\item A user can choose to hide rejected appointments
\end{enumerate} \\ \hline
Testing task &
\begin{task steps}
	\item 1. The user initiate edit or add appointment
	\item 2. A popup window or panel is shown. 
	\item 3. The user select participants from the user- and group-list and invite them.
	\item 4. Login as one of the invited users. 
	\item 5. The appointment is shown in the calendar
	\item 6. Reject the invite and hide the appointment
	\item 7. Login as one of the group members invited
	\item 8. The appointment is shown in the calendar
	\item 9. Accept the invite.
	\item 10. Login as the user in step 3. Check that the updated status from the participants are shown. 
\end{task steps}	\\ \hline
Necessary environmental requirements & More than two users exist, and one group.   \\ \hline
Any dependency between this test and the other tests defined & Login has to work, add or edit appointment  \\ \hline

\end{tabularx}

\subsection{Testcase 4: Edit Appointment}

\begin{tabularx}{\textwidth}{ |X|X| }
\hline
\rowcolor{Gray}
Test ID & 4 \\ \hline
Test item & Edit appointment \\ \hline
Approach & User initiate the edit an appointment action \\ \hline
Item pass/ fail criteria & All of the appointment fields can be edited.\\ \hline
Input data & 
\begin{itemize}
	\item An appointment with participants
\end{itemize}\\ \hline
Expected result & 
\begin{enumerate}
	\item The appointment is updated and all participants are notified
\end{enumerate} \\ \hline
Testing task &
\begin{task steps}
	\item 1. The user initiate edit appointment
	\item 2. A popup window or panel is shown. 
	\item 3. The user edits something.
	\item 4. Login as one of the participants and make sure he is notified. 
\end{task steps}	\\ \hline
Necessary environmental requirements & More than two user exist as well as an appointment is created.   \\ \hline
Any dependent between this test and the other tests defined & Login has to work, add Appointment  \\ \hline

\end{tabularx}

\subsection{Testcase 14: Alarm}

\begin{tabularx}{\textwidth}{ |X|X| }
\hline
\rowcolor{Gray}
Test ID & 14 \\ \hline
Test item & Alarm \\ \hline
Approach & User initiate the add alarm action \\ \hline
Item pass/ fail criteria & The alarm alerts the user in the correct time.\\ \hline
Input data & 
\begin{itemize}
	\item An appointment the user is invited to
	\item Alarm date and time
	\item (Optional) Description
\end{itemize}\\ \hline
Expected result & 
\begin{enumerate}
	\item A alarm is added and executes correctly.
\end{enumerate} \\ \hline
Testing task &
\begin{task steps}
	\item 1. The user initiate edit appointment
	\item 2. A popup window or panel is shown. 
	\item 3. Add alarm that executes in one minute from current time and click save
	\item 5. Wait for the alarm
\end{task steps}	\\ \hline
Necessary environmental requirements & A user exist as well as an appointment is created.   \\ \hline
Any dependent between this test and the other tests defined & Login has to work, add Appointment  \\ \hline

\end{tabularx}

\end{document}
