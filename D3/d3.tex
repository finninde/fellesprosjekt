\documentclass{article}
\usepackage[utf8]{inputenc}
\usepackage[norsk]{babel}
\usepackage{mathtools} 
\usepackage{hyperref}
\usepackage{listings} 
\usepackage{graphicx}


\begin{document}
\begin{titlepage}
\begin{center}

\vspace*{3cm}
\textsc{\Huge MMI - D3}\\[0.7cm]
\textsc{\medium TTM4100 - Communication Services and Networks}\\[0.3cm]
\textsc{\medium TDT4140 - Software Enigneering}\\[0.3cm]
\textsc{\medium TDT4145 - Data Modeling, Databases and Database Management Systems}\\[0.3cm]
\textsc{\medium TDT4180 - Human-Computer Interaction}\\[0.3cm]

\textbf{\Large Gruppe 7:} \\[0.2cm]
\text{\Large Espen Albert, Finn Inderhaug, Kristoffer Andreas Dalby} \\
\text{\Large Christoffer B. NysÊter, Andreas Wien, Jonas André Dalseth}\\[1cm]

\today

\end{center}
\end{titlepage}
\section{Introduksjon}
Denne rapporten omhandler skjermdesignet og konstruksjonen av brukergrensesnittet til gruppe 7 i fellesprosjektet. Vi skal beskrive vår konseptuelle modell, ta for oss skjermdesignet og gi en konstruksjonsbeskrivelse. 

\section{Konseptuell modell}
%Dere skal her beskrive de begrepene brukeren skal forholde seg til i applikasjonen. Bruk UML for å beskrive klassene, datafeltene i klassene, arv, og relasjoner.



\section{Skjermdesign}
%Dere skal her ta utgangspunkt i kravspesifikasjonen og scenariene fra øving D2. I innleveringen skal dere beskrive grafisk struktur og utforming, kobling mot konseptuell modell, og hvordan alle deler av applikasjonen reagerer på relevante hendelser, som museklikk og tastetrykk. Målet er å spesifisere brukeropplevelsen, dvs. hva brukeren til enhver tid ser og kan gjøre.




\section{Konstruksjonsbeskrivelse}
%Mens skjermdesignet fokuserer på brukerens opplevelse, skal dere her beskrive hvordan brukergrensesnittet er bygd opp, dvs. hvilke vinduer og dialogelementer som utgjør grensesnittet, og hvordan disse er koblet sammen i et hierarki og kommuniserer vha. metodekall og hendelser. Her er altså fokuset hvordan konstruksjonselementene fra Swing‐rammeverket benyttes for å realisere brukergrensesnittet. Detaljeringsnivået skal tilsvare formuleringen av konstruksjonsøvingene og i prinsippet gjøre det mulig å skrive en funksjonstest vha. JUnit/JFCUnit‐rammeverket.




\end{document}

