\documentclass[a4paper, english, 12pt]{article}
\usepackage[utf8]{inputenc}
\usepackage[norsk]{babel}
\usepackage{mathtools}
\usepackage{hyperref}
\usepackage{listings}
\usepackage{graphicx}

\begin{document}
\begin{titlepage}
\begin{center} 

\vspace*{3cm}
\textsc{\Huge Fellesprosjektet}\\[0.7cm]
\textsc{\LARGE TTM4100 - Kommunikasjon - Tjenester og nett}\\[0.2cm]
\textsc{\LARGE TDT4140 - Programvareutvikling}\\[0.2cm]
\textsc{\LARGE TDT4145 - Datamodellering og Databasesytemer}\\[0.2cm]
\textsc{\LARGE TDT4180 - Menneske-maskin-interaksjon}\\[0.2cm]
\text{\LARGE Fellesprosjektet}\\[1.6cm]

\textbf{\Large Gruppe 7:} \\[0.2cm]
\text{\Large Espen Albert, Finn Inderhaug, Kristoffer Andreas Dalby} \\
\text{\Large Christoffer B. Nysæter, Andreas Wien, Jonas André Dalseth}\\[1cm] 

\today

\end{center}
\end{titlepage}

\section{Resources}
We have 6 persons available for completing the application. Every person has their own personal computer with  development tools for the \href{http://www.oracle.com/technetwork/java/javase/downloads/java-se-jre-7-download-432155.html}{java runtime environment}, the \href{http://www.postgresql.org/}{postgreSQL} database, 
and the \href{http://json.org/}{JSON} object interface.
The budget for our project is limited to the work hours specified in time estimate.

\section{Time and budget estimate}
Based on the resources we have at our disposal we have made the following time and budget estimates.
Since this project is a school exercise labour is free. Nonetheless we have put together a budget consisting of our work hours and a fictive salary of $849,90 NOK$. Making the salary budget a total of $\approx 734300 NOK$. In addition to creating a calendar, we have also been given a separate project in TTM4100 which we have included here, and will take some of the total work time.
We have estimated a total monetary spending budget of 0 NOK, because we are using free software and the developers personal computers. It is highly unlikely that we will pay for any new software or hardware during this project. 

We have estimated 18 workdays of 8 hours, with 6 developers. This gives us $864$ work hours.

\section{Deadlines}
The absolute deadlines are required from our customer(the exercise) are shown in table \ref{deadline}.

\begin{table}[h]
    \begin{center}
    \caption{Deadlines} 
    \label{deadline}
    \vspace{0,5cm}
    \begin{tabular}{| l | l |}  
        \hline
        Task & Due by date \\
        \hline 
    PU1  Prosjektplan & 2.mars \\
    PU2  Systemtestplan & 2.mars\\
    KTN  Prosjektplan & 3.mars\\
    DB   ER modell& 6.mars\\
    MMI  D2.1 og 2.2 & 7.mars\\
    PU3  Overordnet Design & 9.mars\\
    DB2  Logisk databaseskjema & 14.mars\\
    MMI  del 3 & 14.mars\\
    PU4  Implementasjon og testing & 21.mars\\
    PU5  Dokumentasjon & 21.mars\\
    KTN  working implementation & 24.mars\\
        \hline
    \end{tabular}
    \end{center}
\end{table}



\section{Responsibilities}
We will divide the responsibility for the completion of the exercise in 4 parts; the database, network communication, client model and client view.  There is also a person responsible for human resources, and a union representative chosen democratically by the group. The human resource responsible has taken on him to create a friendly work environment and to plan fun excursions.

\subsection{Database}
The database will hold every piece of information that the program needs to save over a longer time period. 


The database section of the program will be divided as follows and the time estimates will be:

\begin{table}[h]
    \begin{center}
    \caption{Database section and amount of time} 
    \label{database}
    \vspace{0,5cm}
    \begin{tabular}{ll} \\ 
        \hline
        $Task$ & $Estimated hours$\\
        \hline 
    Create ER-diagram for the application & 10 hours\\
    Logic databasechema & 10 hours\\    
    Setup the database server & 8 hours\\
    Create the needed structure in SQL & 16 hours\\
    Implement JDBC & 8 hours\\
    Implement needed methods & 16 hours\\
        \hline
    \end{tabular}
    \end{center}
\end{table}

Total estimated time for implementation of the basic database features will be 68 hours.

\section{Responsibilities}
We will divide the responsibility for the completion of the exercise in 4 parts; the database, network communication, 
client model and client view. 
\subsection{Database}
The database section of the program will be divided as follows:



\subsection{Network communication}
Network communication consists of the part of the server communicating with both clients and the database. It is vital that the server can lock resources, and still queue requests from clients. In table \ref{Network} we have a look at how many work hours that will be allocated to this part of the system.



\subsection{Client model}
The client model can be separated in to several smaller problems that are easier time estimated as parts. The sum of these makes the total time estimate of the clients model as seen in table \ref{clientmodel}. The model will be written in java, and handle json objects from the server. It will also provide a fully fledged API for the GUI\footnote{graphical unit inteface}. 

 \begin{table}[h]
    \begin{center}
    \caption{Client model time estimate} 
    \label{clientmodel}
    \vspace{0,5cm}
    \begin{tabular}{ll} \\ 
        \hline
        Task & Estimated hours\\
        \hline 
	Creating login functionality & 7 \\
	Parsing json objects to model & 8 \\
	Define model and API structure\footnote{Requires a lot of communication with the design responsible}  & 12\\
	Create json objects based on the model & 8 \\
	Creating vital functions and objects & 16 \\
	Create unit tests & 6 \\
	Optimize the notify function\footnote{Both to UI and server} & 15 \\	
        \hline
	Sum & 72\\
	\hline
    \end{tabular}
    \end{center}
\end{table}



\subsection{Client view}
Time to make GUI:

\begin{table}[h]
    \begin{center}
    \caption{Time to design the user interface} 
    \label{UI}
    \vspace{0,5cm}
    \begin{tabular}{ll} \\ 
        \hline
        $Task$ & $Estimated hours$\\
        \hline 
    Discuss how we want the user interface & 12 hours\\
    Designing the user interface using paper models & 12 hours\\    
    Show papermodel to studass and another group & 6 hours\\
    Fix papermodel after feedback & 3 hours\\
    Conceptual model & 6 hours\\
    Screen design & 12 hours\\
    Construction design & 12 hours\\
    Make login screen & 4 hours\\
    Make appointment view & 16 hours\\
    Make week view & 16 hours\\
    Other functionality & 16 hours\\
        \hline
    \end{tabular}
    \end{center}
\end{table}


\end{document}
\subsection{Network communication}
\begin{table}[h]
    \begin{center}
    \caption{Computer networking work breakdown schedule}
    \label{Network}
    \vspace{0,5cm}
    \begin{tabular}{ll} \\
        \hline
        $Task$ & $Subtask & $Estimated hours$ & $ Overall hours\\
        \hline
Planning & Creating a class diagram & 4 & 4 \\
Planning & Sequencediagram for login, send message from client and logout. & 6 & 10 \\
Planning & A short textual description of the design & 2 & 12 \\
Planning & Total & 12 & 12 \\
Implementation and testing & Play with JSON & 6 & 18 \\
Implementation and testing & Server and client login & 6 & 24 \\
Implementation and testing & Server and client logout & 6 & 30 \\
Implementation and testing & Extra functionality & 0-24 & 30 \\
Implementation and testing & System integratition & 10 & 40 \\
Implementation and testing & Total & 28 & 40 \\
Extra testing & Test many clients & 4 & 44 \\
Overall computer networking separate project & Total & 44 \\
----------- & ---------- & -- & -- \\
Planning & Server to client & 6 & 6 \\
Planning & Client to server & 6 & 12 \\
Planning & Total & 12 & 12 \\
Implementation and testing & Server to clients, Master & 10 & 22 \\
Implementation and testing & Server and clients, Threads sending JSON objects to logged in clients & 16 & 28 \\
Implementation and testing & Clients to server, Master & 10 & 38 \\
Implementation and testing & Clients to server, Check for inconsistency & 10 & 48 \\
Implementation and testing & Clients to server, multiple Threads & 10 & 58 \\
Implementation and testing & Total & 46 & 58 \\
Overall computer networking common project & Total & 58 & 58 \\
Overall computer networking & Total & 102 & 102 \\
        \hline
    \end{tabular}
    \end{center}
\end{table}


